\documentclass[10pt]{article}

\usepackage{graphicx}
\usepackage{verbatim}

\title{Exploration of the Amyrel datasets}
\author{Arnaud LE ROUZIC, Jean-Luc DA LAGE}

\begin{document}

\maketitle

\section{Body weight}

\begin{center}
    \includegraphics[width=0.7\textwidth]{../figures/weight-bxp}
\end{center}

Analysis of variance:

\verbatiminput{../figures/weight-aov.txt}

\section{Mortality}

\subsection{Hatching rate}

\begin{center}
    \includegraphics[width=0.7\textwidth]{../figures/mortality-hatched.pdf}
\end{center}

\verbatiminput{../figures/mortality-hatched.txt}

\subsection{Larvae to adult survival}

\begin{center}
    \includegraphics[width=0.7\textwidth]{../figures/mortality-surv.pdf}
\end{center}

\verbatiminput{../figures/mortality-surv.txt}

\section{Competition}

Population genetics model: 

2-allele, 1-locus model ($w$ and $m$ alleles). $p = \mathrm{Freq}(ww) + \frac{1}{2}\mathrm{Freq}(wm)$ is the frequency of $w$ alleles. We assume no dominance; $s$ is the fitness penalty for the $mm$ genotype compared to the wild-type $ww$ genotype. From standard theory: 

$$\Delta p = p \frac{1-\frac{s}{2}(1-p)}{1-s(1-p)}$$. 

The model was fit to the data by maximum likelihood, 2 parameters to estimate ($s$ and $p_0$). The number observed $w$ alleles was expected to follow a binomial distribution, with probability $p_t$ obtained from the expected deterministic dynamics determined by parameters ($s$ and $p_0$) ( = no genetic drift). The model was fit independently on all time series (3 replicates for 2 crosses and 3 media). 

\begin{center}
    \includegraphics[width=\textwidth]{../figures/compet-timeseries.pdf}
\end{center}

\begin{center}
    \includegraphics[width=\textwidth]{../figures/compet-param.pdf}
\end{center}

There was a slight excess of samples in non-Hardy-Weinberg frequencies (36 / 133 for a p-value threshold at 0.05); estimates were sensibly the same when non-Hardy-Weinberg samples were removed.

\begin{minipage}{\linewidth}\verbatiminput{../figures/compet-model.txt}\end{minipage}

\vspace{0.5cm}

When fitted to all time series simultaneously, there was a significant bias in favor of the $w$ allele (about $\sim 2\%$ fitness difference between $ww$ and $mm$). 


\end{document}
